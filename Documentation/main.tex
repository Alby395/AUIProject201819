%++++++++++++++++++++++++++++++++++++++++
% Don't modify this section unless you know what you're doing!
\documentclass[letterpaper,12pt]{article}
\usepackage{tabularx} % extra features for tabular environment
\usepackage{graphicx} % takes care of graphic including machinery
\graphicspath{ {./images/} }
\usepackage[margin=1in,letterpaper]{geometry} % decreases margins
\usepackage{cite} % takes care of citations
\usepackage{float}
\usepackage{textcomp}
\usepackage{amsmath}
%++++++++++++++++++++++++++++++++++++++++


\begin{document}

\begin{titlepage} % Suppresses displaying the page number on the title page and the subsequent page counts as page 1
	\newcommand{\HRule}{\rule{\linewidth}{0.5mm}} % Defines a new command for horizontal lines, change thickness here
	
	\center % Centre everything on the page
	
	%------------------------------------------------
	%	Headings
	%------------------------------------------------
	
	\textsc{\LARGE Advanced User Interface}\\[0.5cm] % Main heading such as the name of your university/college
	
	\textsc{\Large 2018/19}\\[0.5cm] % Major heading such as course name
	
	%------------------------------------------------
	%	Title
	%------------------------------------------------
	
	\HRule\\[0.4cm]
	
	{\huge\bfseries Public speech in Virtual Reality\\SpeechVR}\\[0.4cm] % Title of your document
	
	\HRule\\[1cm]
	
	
	\large\textit{Author}\\[0.3cm]
	Alberto Patti\\
	Yu Zhu\\[1.5cm]
	

	\begin{abstract}
	This project offers people that have fear of speaking in public an instrument that allows them to try their speech in a virtual environment. This basic idea is expanded by using a biosensor to manipulate the environment the user is put in: the amount of people that the user sees in the audience changes based on the anxiety level determined by reading the values from the biosensor.
	\end{abstract}
\end{titlepage}

\tableofcontents
\pagebreak
\section{Introduction}
	\subsection{Purpose}
The purpose of this document is to give information about the "WIVR games for stress relief" project developed for the Advanced User Interface course.\\
This document aims to explain:
\begin{itemize}
	\item The needs, goals and requirements for the targeted users;
	\item Previous researches and projects on the same topic;
	\item The choices made throughout the development of the project;
\end{itemize}

\subsection{Scope}
"Public speech in Virtual Reality" is a VR application that tries to give an instrument to people that have fear of speaking in public to improve their ability to speak to an audience.\\
The application offers a virtual environment where the user can try a speech in front of an audience that can react based on the user performance. The usage of a biosensor allows for the manipulation of the environment so that the user can face up the audience without being too threatened by it. 
\subsection{Definitions, acronyms and abbreviations}
\begin{itemize}
	\item VR: Virtual Reality
	\item HMD: Head Mounted Display
	\item WIVR: Wearable Immersive Virtual Reality
\end{itemize}

	\pagebreak

\section{NGR}
	\subsection{Target}
The main target of the project are people that have fear of speaking in public. This kind of fear can be categorized as part of social phobia, i.e. "persistent fears of situations involving social interaction or social performance or situations in which there is the potential for scrutiny by others"\cite{model}.

\subsection{Context and Needs addressed}
"In social/evaluative situations, the primary threat stimulus is an audience and the primary threatening outcome is negative evaluation from the audience"\cite{model}.
The idea of being evaluated by the audience is enough to start a loop that keeps fueling the anxiety of the subject as shown in figure \ref{fig:model}.
\begin{figure}[!h]
	\centering
	\includegraphics[scale=0.55]{Model}
	\caption{A model of the generation and maintenance of anxiety in social/evaluative situations\cite{model}.}\label{fig:model}
\end{figure}

For this reason the needs that were formulated are:
\begin{itemize}
	\item Have more confidence around people during the speech
	\item Listen to the speech after the performance
\end{itemize}

\subsection{Constraints}
\begin{itemize}
	\item HMD;
	\item A smartphone running Android Jellybean or higher (4.1.x+);
	\item Empatica E4;
	\item Microphone;
	\item Headphones;
	\item A pc (Windows 7 or higher) with Visual C++ Redistributable Package installed
	\item Bluegiga Bluetooth Smart Dongle
	\item Comfortable place where the user can sit down and rest the arm
\end{itemize}

\subsection{Goals}
\begin{itemize}
	\item Improve the ability to speak in public
	\item Allow the subject to be less anxious before and during the speech
\end{itemize}

\subsection{Requirements}

\begin{itemize}
	\item The applications should provide an environment where the subject can try his/her speech in front of a virtual audience.
	\item The application should progressively change the number of people that the user can see in the audience based on his/her state of mind.
	\item The application should calm the subject if needed.
	\item The application must stop the test in case the subject doesn't feel well.
	\item The application should reward the user on a good performance.
	\item The application should record the speech and play it if needed.
\end{itemize}





	\pagebreak
	
\section{State of the art}
	\subsection{Applications}
There are many application with the same objective that were developed and are nowadays available:
\begin{itemize}
	\item Virtual Orator
	\item Speech Center VR
	\item VirtualSpeech
	\item \#BeFearless
	\item Public Speaking Simulator VR
\end{itemize}
Some of them are available on smartphone other are only available on PC using a HDM (HTC Vive, Oculus Rift etc...). All of them offer similar features but they also offer unique options to differentiate from the others application.
{
\renewcommand{\arraystretch}{1.5}
\begin{table}[h]
	\centering
	\begin{tabular}{l|c|c|c|c|c|c|c|c|c|c|}
		\cline{2-11}
 		& \multicolumn{1}{l|}{\rotatebox{270}{Multiple Environment}} & \multicolumn{1}{l|}{\rotatebox{270}{Upload documents}} & \multicolumn{1}{l|}{\rotatebox{270}{Record your performance}}
 		& \multicolumn{1}{l|}{\rotatebox{270}{Question from the audience}} & \multicolumn{1}{l|}{\rotatebox{270}{Speech analysis}} & \multicolumn{1}{l|}{\rotatebox{270}{Distractions}}
 		& \multicolumn{1}{l|}{\rotatebox{270}{during the speech} \newline \rotatebox{270}{Variable number of people} } & \multicolumn{1}{l|}{\rotatebox{270}{Biosensor}}
 		& \multicolumn{1}{l|}{\rotatebox{270}{Lectures}} & \multicolumn{1}{l|}{\rotatebox{270}{Evaluation of the performance }} \\ \hline
		
		\multicolumn{1}{|l|}{Virtual Orator} & X & X & X & X &  & X &  &  &  &  \\ \hline
		\multicolumn{1}{|l|}{Speech Center VR} & X & X & X &  &  & X &  &  & X & X \\ \hline
		\multicolumn{1}{|l|}{VirtualSpeech} & X & X & X &  & X & X &  & X & X & X \\ \hline
		\multicolumn{1}{|l|}{\#BeFearless} & X & X & X &  & X &  &  & X &  & X \\ \hline
		\multicolumn{1}{|l|}{Public Speaking Simulator VR} &  &  &  &  &  & X & X &  &  &  \\ \hline
	\end{tabular}
\end{table}
}

This project uses the same general idea as these applications and tries to expand it by introducing a biosensor as a mean to change the virtual environment the user is put in. 

\subsection{Research}
There are many researches about public speech anxiety (and social phobia) but the most relevant for the sake of this project are:

\begin{itemize}
	\item Slater, M., Pertaub, D. P., \& Steed, A. (1999). Public speaking in virtual reality: Facing an audience of avatars. IEEE Computer Graphics and Applications, 19(2), 6-9.\\[0.15cm]
	The focus of this paper is to analyze how people evaluate themselves while in front of an audience with different reactions using VR.
	
	\item Pertaub, D. P., Slater, M., \& Barker, C. (2002). An experiment on public speaking anxiety in response to three different types of virtual audience. Presence: Teleoperators \& Virtual Environments, 11(1), 68-78.\\[0.15cm]
	This is an extension of the previous research.
	
	\item Chollet, M., Sratou, G., Shapiro, A., Morency, L. P., \& Scherer, S. (2014, May). An interactive virtual audience platform for public speaking training. In Proceedings of the 2014 international conference on Autonomous agents and multi-agent systems (pp. 1657-1658). International Foundation for Autonomous Agents and Multiagent Systems.\\[0.15cm]
	The focus of this research is to design a way to let people learn how to behave in front of a fake audience that reacts to the user actions. This research doesn't use VR but instead works with screens and audiovisual sensors to analyze the user behaviour.
	
	\item Poeschl, S., \& Doering, N. (2012, March). Virtual training for Fear of Public Speaking—Design of an audience for immersive virtual environments. In Virtual Reality Short Papers and Posters (VRW), 2012 IEEE (pp. 101-102). IEEE.\\[0.15cm]
	This research explains how to develop an audience that shows realistic behaviour.
	
	\item McKinney, M. E., Gatchel, R. J., \& Paulus, P. B. (1983). The effects of audience size on high and low speech-anxious subjects during an actual speaking task. Basic and Applied Social Psychology, 4(1), 73-87.\\[0.15cm]
	This research studies how people react during a speech in front of different amount of people hearing.
\end{itemize}
	\pagebreak
	
\section{UX design}
	\subsection{General}
The application is divided into two modes: free talk and interview. The former is for whom have their own speech material, while the later without material will be interviewed by other users. The audience are divided in three different characters: kind classmates \& colleagues, serious experts and indifferent people. Different character will give different feedback during the speech. The user could choose the number and composition of the audience.

\subsection{Audience}
\begin{itemize}
	\item Kind classmates \& colleagues: with causal wearing and friendly smile, they will listen with smile, nod and praise during the speech.
	\item Indifferent people: with strange wearing, they pay no attention to the speaker and will look around, speak to others and fall asleep during the speech.
	\item Serious experts: with formal suit and serious expression, they will listen without emotion, shake head and get angry during the speech.
\end{itemize}

\begin{figure}[H]
	\centering
	\includegraphics[scale=0.5]{Audience}
	\caption{Audience characteristics}
\end{figure}

\subsection{Audience number and composition}
\begin{itemize}
	\item Number: the number of audience will be changed by the light in the theater. More audience will come into view with bigger light.
	\item Composition: the users can choose different compositions of audience in different characters, which will help them to meet their own level.
\end{itemize}

\begin{figure}[H]
	\centering
	\includegraphics[scale=0.528]{Composition}
	\caption{Number and Composition}
\end{figure}

\subsection{Process}
\begin{figure}[H]
	\centering
	\includegraphics[scale=0.434]{Process}
	\caption{Flow of the application}
\end{figure}


\subsection{Interface}
\begin{figure}[H]
	\centering
	\includegraphics[scale=0.521]{Interface}
	\caption{Stage 1: App}
\end{figure}

\begin{figure}[H]
	\centering
	\includegraphics[scale=0.45]{Breathing}
	\caption{Stage 2: Heart rate test}
\end{figure}
	\pagebreak
	
\section{Implementation}
	\subsection{Introduction}
The main application of the project is an Android app built on Unity. This allows the creation of a VR environment with ease. The only problem that arises from this choice is that it isn't possible to retrieve the data from the biosensor and send them to the smartphone directly as Unity doesn't allow a direct communication. As shown in figure \ref{fig:communication} the information from the biosensor are read first by a Computer and then sent to a Firebase server that stores the values. This values are then read by the Android application using a HTTP request.
\begin{figure}[h]
	\centering
	\includegraphics[scale=0.7]{ConnectionDiagram}
	\caption{Diagram that shows how the communication from the biosensor to the smartphone works.}\label{fig:communication}
\end{figure}

\subsection{Android Application}
Language used: C\#\\
Plugins:
\begin{itemize}
	\item ZXing
	\item Android Runtime Permissions
\end{itemize}

\subsection{Computer Client}
Language used: Java\\
Plugins:
\begin{itemize}
	\item ZXing
	\item JavaFX
\end{itemize}

\subsubsection{Description}
The computer client main task is to retrieve data from the Empatica E4 and send the values to the Firebase server. In order to do this, it communicates with E4 streaming server, an application that allows to forward realtime data of multiple Empatica E4 devices to multiple TCP socket connections.\\
The E4 Streaming server works through a message protocol where client request are in the following format:
\begin{center}
	COMMAND ARGUMENT\_LIST
\end{center}
Messages from server containing responses to commands are in the following format
\begin{center}
	COMMAND ARGUMENT\_LIST
\end{center}
Messages from server containing data from device are in the following format
\begin{center}
	STREAM\_TYPE TIMESTAMP DATA
\end{center}
The commands used from the client are:
\begin{itemize}
	\item device\_list\\
	requests the list of Empatica E4 devices to the E4 Streaming server
	\item device\_connect DEVICE\_ID\\
	sends a connection request to a specific device
	\item device\_subscribe STREAM STATUS\\
	start or stop receiving data from a given stream.
	\item device\_disconnect\\
	sends a device disconnection request
\end{itemize}
\pagebreak
\subsubsection{Algorithm design}
\begin{figure}[H]
	\centering
	\includegraphics[scale=0.5]{BiosensorConnectDiagram}
	\caption{Flowchart that describes how the connection to the Empatica E4 device works}
\end{figure}

\begin{figure}[H]
	\centering
	\includegraphics[scale=0.55]{BiosensorDataDiagram}
	\caption{Flowchart that describes how the client retrieve data from the Empatica E4}
\end{figure}


	\pagebreak
	
\section{Value proposition}
	\subsection{Challenges}
The main challenges that came up during the development of the project were:
\begin{itemize}
	\item finding a way to put the subject in a "controlled" stressful situation without leaving him/her in an anxious state;
	\item giving instruction to the user so that he/she won't stay silent while in front of the virtual audience.
\end{itemize}


\subsection{Main difficulties}
\begin{itemize}
	\item Even though Android devices are able to run VR application, they can only be used to display simple scenery, games with a limited amount of polygons or 360\textdegree{} videos. In fact, the limited resources available on a smartphone makes it difficult to develop VR applications that can run without performance problems. Because of this, during development some choices had to be adjusted or changed in order to make the application run. Unfortunately, this was not enough. In fact, the smartphone used to test the application (Huawei P10 lite) was not powerful enough to handle it correctly.\\
	\item The biosensor used (Empatica E4) isn't the best fit for the purpose of this project. It is able to track data in real time but even the smallest movement is enough to disrupt the readings, leading to either wrong values or no value at all. Also, the Empatica E4 doesn't track the HR directly, it needs to take the Inter-Beat Interval (ibi) and convert it into HR.
	\begin{equation}
		HR = \left \lfloor \frac{60}{ibi} \right \rfloor
	\end{equation}
\end{itemize}

\subsection{Analysis}
The effectiveness of VR as a mean to ease the anxiety of people that have the fear of speaking in public is a fact reported in many researches: it doesn't really improve the people's ability to talk to an audience but it helps them gaining enough self confidence so that they can gather the courage to face the audience.\\
As stated previously, there are many applications that allows the user to deal with this kind of fear but none of them uses a biosensor as a mean to manage the environment, instead they use it as a way to measure the overall score of the performance. This is what makes this project different: even though the base application is the same as the other, it offers an experience that changes based on the values read from the biosensor.
	\pagebreak

\section{Future work}
	In the future this project could be extended in many different ways:
\begin{itemize}
	\item The Empatica E4 biosensor could be changed with a more specialized one that could track parameters in a more precise way. One option could be the usage of smart clothing that offer the possibility to track both HR and GSR while also allowing the tracking of the breathing activity and many other parameters. This choice would also allow to have more precise data that should be less influenced by the movement compared to the Empatica E4.
	\item As stated before, the application is not well optimized to run on an Android device, in fact it runs at less than the required 60 FPS. Unity released a new feature with Unity 2018.3: lightweight rendering pipeline (LW RP). As reported on the Unity blog, "The goal of the LW RP is to provide optimized real time performance on performance constrained platforms by making some tradeoffs with regard to lighting and shading."\footnote{https://blogs.unity3d.com/2018/02/21/the-lightweight-render-pipeline-optimizing-real-time-performance/}.\\
	This feature is certainly not enough to achieve a good result but it may be a starting point to improve the overall performance of the application.
	\item All the algorithm related to the biosensor and microphone were made just to show the general functionality of this project and so they are really simple. In the future, these could be improved in order to have a better control over the environment and the situation the user is put in.
\end{itemize}
	\pagebreak


%++++++++++++++++++++++++++++++++++++++++
% References section will be created automatically 
% with inclusion of "thebibliography" environment
% as it shown below. See text starting with line
% \begin{thebibliography}{99}
% Note: with this approach it is YOUR responsibility to put them in order
% of appearance.

\begin{thebibliography}{99}

\bibitem{model}
Rapee, R. M., \& Heimberg, R. G. (1997). A cognitive-behavioral model of anxiety in social phobia. Behaviour research and therapy, 35(8), 741-756.

\bibitem{VRPublicSpeaking}
Slater, M., Pertaub, D. P., \& Steed, A. (1999). Public speaking in virtual reality: Facing an audience of avatars. IEEE Computer Graphics and Applications, 19(2), 6-9.

\bibitem{VRPublicSpeaking2}
Pertaub, D. P., Slater, M., \& Barker, C. (2002). An experiment on public speaking anxiety in response to three different types of virtual audience. Presence: Teleoperators \& Virtual Environments, 11(1), 68-78.

\bibitem{VRPublicSpeaking3}
Chollet, M., Sratou, G., Shapiro, A., Morency, L. P., \& Scherer, S. (2014, May). An interactive virtual audience platform for public speaking training. In Proceedings of the 2014 international conference on Autonomous agents and multi-agent systems (pp. 1657-1658). International Foundation for Autonomous Agents and Multiagent Systems.

\bibitem{VRPublicSpeaking4}
Poeschl, S., \& Doering, N. (2012, March). Virtual training for Fear of Public Speaking—Design of an audience for immersive virtual environments. In Virtual Reality Short Papers and Posters (VRW), 2012 IEEE (pp. 101-102). IEEE.

\bibitem{VRPublicSpeaking5}
McKinney, M. E., Gatchel, R. J., \& Paulus, P. B. (1983). The effects of audience size on high and low speech-anxious subjects during an actual speaking task. Basic and Applied Social Psychology, 4(1), 73-87.

\bibitem{breathing}
Bernardi, L., Wdowczyk-Szulc, J., Valenti, C., Castoldi, S., Passino, C., Spadacini, G., \& Sleight, P. (2000). Effects of controlled breathing, mental activity and mental stress with or without verbalization on heart rate variability. Journal of the American College of Cardiology, 35(6), 1462-1469.
\end{thebibliography}


\end{document}