\subsection{Purpose}
The purpose of this document is to give information about the "WIVR games for stress relief" project developed for the Advanced User Interface course.\\
This document aims to explain:
\begin{itemize}
	\item The needs, goals and requirements for the targeted users;
	\item Previous researches and projects on the same topic;
	\item The choices made throughout the development of the project;
\end{itemize}

\subsection{Scope}
"Public speech in Virtual Reality" (SpeechVR) is a VR application that tries to give an instrument to people that have fear of speaking in public to improve their ability to speak to an audience.\\
The application offers to the user a virtual theatre where he/she can try a speech in front of an audience that can react based on his/her performance. The main functionality of the application is given by a biosensor (Empatica E4) that allows the tracking of the heart rate and the galvanic skin response of the user to evaluate the state of mind of the subject and decide how the environment should change: whether thee amount of people that the user sees in the audience can be changed or, in case the application consider that the user is in a situation of high stress, block the test.
 
\subsection{Definitions, acronyms and abbreviations}
\begin{itemize}
	\item VR: Virtual Reality
	\item HMD: Head Mounted Display
	\item WIVR: Wearable Immersive Virtual Reality
	\item HR: Heart Rate
	\item GSR: Galvanic Skin Response
\end{itemize}
