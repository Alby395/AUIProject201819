\subsection{Applications}
There are many application with the same objective that were developed and are nowadays available:
\begin{itemize}
	\item Virtual Orator
	\item Speech Center VR
	\item VirtualSpeech
	\item \#BeFearless
	\item Public Speaking Simulator VR
\end{itemize}
Some of them are available on smartphone other are only available on PC using a HDM (HTC Vive, Oculus Rift etc...). All of them offer similar features but they also offer unique options to differentiate from the others application.
{
\renewcommand{\arraystretch}{1.5}
\begin{table}[h]
	\centering
	\begin{tabular}{l|c|c|c|c|c|c|c|c|c|c|}
		\cline{2-11}
 		& \multicolumn{1}{l|}{\rotatebox{270}{Multiple Environment}} & \multicolumn{1}{l|}{\rotatebox{270}{Upload documents}} & \multicolumn{1}{l|}{\rotatebox{270}{Record your performance}}
 		& \multicolumn{1}{l|}{\rotatebox{270}{Question from the audience}} & \multicolumn{1}{l|}{\rotatebox{270}{Speech analysis}} & \multicolumn{1}{l|}{\rotatebox{270}{Distractions}}
 		& \multicolumn{1}{l|}{\rotatebox{270}{during the speech} \newline \rotatebox{270}{Variable number of people} } & \multicolumn{1}{l|}{\rotatebox{270}{Biosensor}}
 		& \multicolumn{1}{l|}{\rotatebox{270}{Lectures}} & \multicolumn{1}{l|}{\rotatebox{270}{Evaluation of the performance }} \\ \hline
		
		\multicolumn{1}{|l|}{Virtual Orator} & X & X & X & X &  & X &  &  &  &  \\ \hline
		\multicolumn{1}{|l|}{Speech Center VR} & X & X & X &  &  & X &  &  & X & X \\ \hline
		\multicolumn{1}{|l|}{VirtualSpeech} & X & X & X &  & X & X &  & X & X & X \\ \hline
		\multicolumn{1}{|l|}{\#BeFearless} & X & X & X &  & X &  &  & X &  & X \\ \hline
		\multicolumn{1}{|l|}{Public Speaking Simulator VR} &  &  &  &  &  & X & X &  &  &  \\ \hline
	\end{tabular}
\end{table}
}

This project uses the same general idea as these applications and tries to expand it by introducing a biosensor as a mean to change the virtual environment the user is put in. 

\subsection{Research}
There are many researches about public speech anxiety (and social phobia) but the most relevant for the sake of this project are:

\begin{itemize}
	\item Slater, M., Pertaub, D. P., \& Steed, A. (1999). Public speaking in virtual reality: Facing an audience of avatars. IEEE Computer Graphics and Applications, 19(2), 6-9.\\[0.15cm]
	The focus of this paper is to analyze how people evaluate themselves while in front of an audience with different reactions using VR.
	
	\item Pertaub, D. P., Slater, M., \& Barker, C. (2002). An experiment on public speaking anxiety in response to three different types of virtual audience. Presence: Teleoperators \& Virtual Environments, 11(1), 68-78.\\[0.15cm]
	This is an extension of the previous research.
	
	\item Chollet, M., Sratou, G., Shapiro, A., Morency, L. P., \& Scherer, S. (2014, May). An interactive virtual audience platform for public speaking training. In Proceedings of the 2014 international conference on Autonomous agents and multi-agent systems (pp. 1657-1658). International Foundation for Autonomous Agents and Multiagent Systems.\\[0.15cm]
	The focus of this research is to design a way to let people learn how to behave in front of a fake audience that reacts to the user actions. This research doesn't use VR but instead works with screens and audiovisual sensors to analyze the user behaviour.
	
	\item Poeschl, S., \& Doering, N. (2012, March). Virtual training for Fear of Public Speaking—Design of an audience for immersive virtual environments. In Virtual Reality Short Papers and Posters (VRW), 2012 IEEE (pp. 101-102). IEEE.\\[0.15cm]
	This research explains how to develop an audience that shows realistic behaviour.
	
	\item McKinney, M. E., Gatchel, R. J., \& Paulus, P. B. (1983). The effects of audience size on high and low speech-anxious subjects during an actual speaking task. Basic and Applied Social Psychology, 4(1), 73-87.\\[0.15cm]
	This research studies how people react during a speech in front of different amount of people hearing.
\end{itemize}