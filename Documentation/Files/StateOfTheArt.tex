\subsection{Applications}
There are many application with the same target and objectives as this project that were developed and are nowadays available:
\begin{itemize}
	\item Virtual Orator
	\item Speech Center VR
	\item VirtualSpeech
	\item \#BeFearless
	\item Public Speaking Simulator VR
\end{itemize}
These are some of the available applications but, even though the premises are the same, each of them present different features and are available on different platform as shown in the tables below.

{
\renewcommand{\arraystretch}{1.3}
\begin{table}[H]
	\centering
	\begin{tabular}{|l|l|l|}
		\hline
		Virtual Orator & Oculus Rift / HTC Vive & 3D Environment\\ \hline
		Speech Center VR & Oculus Rift & 3D Environment\\ \hline
		VirtualSpeech & Android & 360° video\\ \hline
		\#BeFearless & Android & 3D Environment\\ \hline
		Public Speaking Simulator VR & Android & 3D Environment\\ \hline
	\end{tabular}
\end{table}
} 
{
\renewcommand{\arraystretch}{1.5}
\begin{table}[H]
	\centering
	\begin{tabular}{l|c|c|c|c|c|c|c|c|c|c|}
		\cline{2-11}
 		& \multicolumn{1}{l|}{\rotatebox{270}{Multiple Environment}} & \multicolumn{1}{l|}{\rotatebox{270}{Upload documents}} & \multicolumn{1}{l|}{\rotatebox{270}{Record your performance}}
 		& \multicolumn{1}{l|}{\rotatebox{270}{Question from the audience}} & \multicolumn{1}{l|}{\rotatebox{270}{Speech analysis}} & \multicolumn{1}{l|}{\rotatebox{270}{Distractions}}
 		& \multicolumn{1}{l|}{\rotatebox{270}{during the speech} \newline \rotatebox{270}{Variable number of people} } & \multicolumn{1}{l|}{\rotatebox{270}{Biosensor}}
 		& \multicolumn{1}{l|}{\rotatebox{270}{Lectures}} & \multicolumn{1}{l|}{\rotatebox{270}{Evaluation of the performance }} \\ \hline
		
		\multicolumn{1}{|l|}{Virtual Orator} & X & X & X & X &  & X &  &  &  &  \\ \hline
		\multicolumn{1}{|l|}{Speech Center VR} & X & X & X &  &  & X &  &  & X & X \\ \hline
		\multicolumn{1}{|l|}{VirtualSpeech} & X & X & X &  & X & X &  & X & X & X \\ \hline
		\multicolumn{1}{|l|}{\#BeFearless} & X & X & X &  & X &  &  & X &  & X \\ \hline
		\multicolumn{1}{|l|}{Public Speaking Simulator VR} &  &  &  &  &  & X & X &  &  &  \\ \hline
	\end{tabular}
\end{table}
}
The base of this project is the same as the application listed before: giving the user an environment where he/she can freely try his/her speech. What makes SpeechVR unique is the usage of a biosensor as a mean to control the environment the user is put in. In fact, the only app that uses a biosensor are VirtualSpeech and \#BeFearless but they use it as another parameter to give a score to the overall performance.

\subsection{Research}
There are many researches about public speech anxiety (and social phobia in general) but the most relevant for the sake of this project are:

\begin{itemize}
	\item Slater, M., Pertaub, D. P., \& Steed, A. (1999). Public speaking in virtual reality: Facing an audience of avatars.\cite{VRPublicSpeaking}\\[0.15cm]
	The focus of this paper is to analyze how people evaluate themselves while in front of an audience with different reactions using VR.
	
	\item Pertaub, D. P., Slater, M., \& Barker, C. (2002). An experiment on public speaking anxiety in response to three different types of virtual audience.\cite{VRPublicSpeaking2}\\[0.15cm]
	This is a more thorough analysis of the previous research.
	
	\item Chollet, M., Sratou, G., Shapiro, A., Morency, L. P., \& Scherer, S. (2014, May). An interactive virtual audience platform for public speaking training.\cite{VRPublicSpeaking3} \\[0.15cm]
	The focus of this research is to design a way to let people learn how to behave in front of a fake audience that reacts to the user actions. This research doesn't use VR but instead works with screens and audiovisual sensors (Kinect) to analyze the user behaviour.
	
	\item Poeschl, S., \& Doering, N. (2012, March). Virtual training for Fear of Public Speaking—Design of an audience for immersive virtual environments.\cite{VRPublicSpeaking4}\\[0.15cm]
	This research explains how to develop an audience that shows realistic behaviour.
	
	\item McKinney, M. E., Gatchel, R. J., \& Paulus, P. B. (1983). The effects of audience size on high and low speech-anxious subjects during an actual speaking task.\cite{VRPublicSpeaking5}\\[0.15cm]
	This research studies how people react during a speech in front of different amount of people hearing.
\end{itemize}