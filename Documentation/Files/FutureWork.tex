In the future this project could be extended in many different ways:
\begin{itemize}
	\item The Empatica E4 biosensor could be changed with a more specialized one that could track parameters in a more precise way. One option could be the usage of smart clothing that offer the possibility to track both HR and GSR while also allowing the tracking of the breathing activity and many other parameters. This choice would also allow to have more precise data that should be less influenced by the movement compared to the Empatica E4.
	\item As stated before, the application is not well optimized to run on an Android device, in fact it runs at less than the required 60 FPS. Unity released a new feature with Unity 2018.3: lightweight rendering pipeline (LW RP). As reported on the Unity blog, "The goal of the LW RP is to provide optimized real time performance on performance constrained platforms by making some tradeoffs with regard to lighting and shading."\footnote{https://blogs.unity3d.com/2018/02/21/the-lightweight-render-pipeline-optimizing-real-time-performance/}.\\
	This feature is certainly not enough to achieve a good result but it may be a starting point to improve the overall performance of the application.
	\item All the algorithm related to the biosensor and microphone were made just to show the general functionality of this project and so they are really simple. In the future, these could be improved in order to have a better control over the environment and the situation the user is put in.
\end{itemize}