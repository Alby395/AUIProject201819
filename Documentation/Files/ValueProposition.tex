\subsection{Challenge}
The main challenge that came up during the development of the project were:
\begin{itemize}
	\item finding a way to put the subject in a "controlled" stressful situation without leaving him/her in an anxious state;
	\item give instruction to the user so that he/she won't stay silent while in front of the virtual audience.
\end{itemize}


\subsection{Main difficulties}
The main difficulty of this project is given by the platform used. Even though Android devices are able to run VR application, they have really limited resources.In fact, they can be used to display simple scenery or games with a limited amount of polygons or 360\textdegree{} videos. Because of this limitation, some choices had to adjusted making the development harder.\\
Also, the biosensor used (Empatica E4) isn't the best fit for the purpose of this project. It can track the necessary data but it wasn't intended as a device to track parameters in real time and so noise and other problems were introduced.